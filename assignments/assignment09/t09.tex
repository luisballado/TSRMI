%%%%%%%%%%%%%%%%%%%%%%%%%%%%%%%%%%%%%%%%%
% Lachaise Assignment
% LaTeX Template
% Version 1.0 (26/6/2018)
%
% This template originates from:
% http://www.LaTeXTemplates.com
%
% Authors:
% Marion Lachaise & François Févotte
% Vel (vel@LaTeXTemplates.com)
%
% License:
% CC BY-NC-SA 3.0 (http://creativecommons.org/licenses/by-nc-sa/3.0/)
% 
%%%%%%%%%%%%%%%%%%%%%%%%%%%%%%%%%%%%%%%%%

%----------------------------------------------------------------------------------------
%	PACKAGES AND OTHER DOCUMENT CONFIGURATIONS
%----------------------------------------------------------------------------------------

\documentclass{article}

\input{structure.tex} % Include the file specifying the document structure and custom commands

%----------------------------------------------------------------------------------------
%	ASSIGNMENT INFORMATION
%----------------------------------------------------------------------------------------

\title{TSRMI: Assignment \#9} % Title of the assignment

\author{Luis Alberto Ballado Aradias\\ \texttt{luis.ballado@cinvestav.mx}} % Author name and email address

\date{CINVESTAV UNIDAD TAMAULIPAS --- \today} % University, school and/or department name(s) and a date

%----------------------------------------------------------------------------------------

\begin{document}

\maketitle % Print the title

%----------------------------------------------------------------------------------------
%	INTRODUCTION
%----------------------------------------------------------------------------------------


Potencialmente, el número de aristas que pueden aparecer en un grafo de visibilidad es $n^{2}$, con n el número total de vértices en el medio ambiente considerado. Para reducir el número de aristas (y reducir el espacio en memoria requerido para el almacenamiento del grafo), sin perder la solución óptima, se propuso una construcción alternativa denominada grafo de cotangentes o grafo-T. \\\\
Documentar algún desarrollo científico que emplee esta representación.

\begin{itemize}
  \item ¿Cuál es el principio de construcción de esta representación?
  \item ¿Cuál es el algoritmo empleado para su construcción?
\end{itemize}

%\begin{figure}[h]
%\includegraphics[width=10cm]{images/vant.jpg}
%\centering
%\end{figure}

\end{document}
