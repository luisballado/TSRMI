%%%%%%%%%%%%%%%%%%%%%%%%%%%%%%%%%%%%%%%%%
% Lachaise Assignment
% LaTeX Template
% Version 1.0 (26/6/2018)
%
% This template originates from:
% http://www.LaTeXTemplates.com
%
% Authors:
% Marion Lachaise & François Févotte
% Vel (vel@LaTeXTemplates.com)
%
% License:
% CC BY-NC-SA 3.0 (http://creativecommons.org/licenses/by-nc-sa/3.0/)
% 
%%%%%%%%%%%%%%%%%%%%%%%%%%%%%%%%%%%%%%%%%

%----------------------------------------------------------------------------------------
%	PACKAGES AND OTHER DOCUMENT CONFIGURATIONS
%----------------------------------------------------------------------------------------

\documentclass{article}

\input{structure.tex} % Include the file specifying the document structure and custom commands

%----------------------------------------------------------------------------------------
%	ASSIGNMENT INFORMATION
%----------------------------------------------------------------------------------------

\title{TSRMI: Assignment \#19} % Title of the assignment

\author{Luis Alberto Ballado Aradias\\ \texttt{luis.ballado@cinvestav.mx}} % Author name and email address

\date{CINVESTAV UNIDAD TAMAULIPAS --- \today} % University, school and/or department name(s) and a date

%----------------------------------------------------------------------------------------

\begin{document}

\maketitle % Print the title

%----------------------------------------------------------------------------------------
%	INTRODUCTION
%----------------------------------------------------------------------------------------

En el ejemplo de las notas de clase, el modelo del sensor no está completo. 

\begin{itemize}
\item ¿Por qué las probabilidades dadas p(bright$|$home) y p(bright$|\neg$ home) no suman 1? \\\\
  El modelo del sensor no está completo porque las probabilidades dadas p(bright|home) y p(bright|$\neg$home) no suman 1. Esto puede deberse a varias razones, como la presencia de ruido en las mediciones o la falta de información completa sobre el comportamiento del sensor.\\

  Para que las probabilidades de observación dadas sean un modelo de sensor válido, deben cumplir con la propiedad de normalización, es decir, la suma de todas las probabilidades condicionales de una observación debe ser igual a 1. %Sin embargo, en este caso, las probabilidades p(bright|home) = 0.6 y p(bright|$\neg$home) = 0.3 no suman 1.\\

  %El modelo completo del sensor debe especificar las probabilidades condicionales para todas las combinaciones posibles de valores de X (ubicación del VANT) y Z (observación del sensor). En este caso, se requieren las probabilidades p(Z|X) para p(bright|home) y p(bright|$\neg$home).\\

  %Dado que las probabilidades dadas no suman 1, es necesario encontrar los valores faltantes para completar el modelo del sensor. Sin esta información adicional, no es posible determinar el valor de p($\neg$home|bright), que representa la probabilidad de que el VANT no se encuentre encima de la estación de aterrizaje dado que se observa luz brillante.\\
  
\item ¿Cuál es el modelo completo del sensor?\\

  p(bright|home) = 1 (El sensor detecta la luz cuando el VANT está sobre la estación de aterrizaje)\\
  p(bright|$\neg$home) = 0 (El sensor no detecta la luz cuando el VANT no está sobre la estación)\\

  %Con este modelo completo del sensor, podemos calcular la probabilidad de que el VANT no esté sobre la estación de aterrizaje dado que se observa luz brillante p($\neg$home|bright) utilizando el teorema de Bayes:\\
%\[
%p(\neg home|bright)= \frac{p(bright|\neg home)*p(\neg home)}{p(bright)}\\
%\]

%Sustituyendo los valores conocidos:\\

%\[
%p(\neg home|bright) = \frac{0 * (1 - 0.5)}{p(bright)}
%\]

%Como $p(bright) = p(bright|home)*p(home)+p(bright|\neg home)*p(\neg home)$, podemos calcular:\\

%$p(bright) = 1 * 0.5 + 0 * 0.5 = 0.5$\\

%Por lo tanto, podemos obtener el valor de p($\neg$home|bright):\\

%\[
%p(\neg home|bright) = \frac{0 * (1 - 0.5)}{0.5} = 0
%\]

\item ¿Cuál es el valor de p($\neg$home|bright)?\\

  Para calcular el valor de p($\neg$home|bright), necesitamos aplicar el teorema de Bayes y utilizar el modelo completo del sensor:\\
\[
p(\neg home|bright) = \frac{p(bright|\neg home)*p(\neg home) }{p(bright)}
\]

Según el modelo completo del sensor que hemos establecido previamente:\\

p(bright|home) = 1\\
p(bright|$\neg$ home) = 0\\

También se proporciona la información de que p(X = home) = 0.5, lo que implica que \\
$p(X=\neg home) = 1 - p(X = home) = 0.5$\\

Ahora, calculemos p(bright):\\
\[
p(bright) = p(bright|home)*p(home)+p(bright|\neg home)*p(\neg home) = 1 * 0.5 + 0 * 0.5 = 0.5\\
\]
Finalmente, sustituimos estos valores en la fórmula de Bayes:\\

\[
p(\neg home|bright) = \frac{p(bright|\neg home)*p(\neg home)}{p(bright)}\\
= \frac{0 * 0.5}{0.5}\\
= \frac{0}{0.5}\\
= 0\\
\]

%Por lo tanto, el valor de p(¬home | bright) es 0, lo que significa que la probabilidad de que el VANT no esté sobre la estación de aterrizaje dado que se observa luz brillante es 0. En otras palabras, cuando se detecta luz brillante, la probabilidad de que el VANT esté sobre la estación de aterrizaje es 1 (100\%).
 
\end{itemize}

\end{document}
