%%%%%%%%%%%%%%%%%%%%%%%%%%%%%%%%%%%%%%%%%
% Lachaise Assignment
% LaTeX Template
% Version 1.0 (26/6/2018)
%
% This template originates from:
% http://www.LaTeXTemplates.com
%
% Authors:
% Marion Lachaise & François Févotte
% Vel (vel@LaTeXTemplates.com)
%
% License:
% CC BY-NC-SA 3.0 (http://creativecommons.org/licenses/by-nc-sa/3.0/)
% 
%%%%%%%%%%%%%%%%%%%%%%%%%%%%%%%%%%%%%%%%%

%----------------------------------------------------------------------------------------
%	PACKAGES AND OTHER DOCUMENT CONFIGURATIONS
%----------------------------------------------------------------------------------------

\documentclass{article}

\input{structure.tex} % Include the file specifying the document structure and custom commands

%----------------------------------------------------------------------------------------
%	ASSIGNMENT INFORMATION
%----------------------------------------------------------------------------------------

\title{TSRMI: Assignment \#13} % Title of the assignment

\author{Luis Alberto Ballado Aradias\\ \texttt{luis.ballado@cinvestav.mx}} % Author name and email address

\date{CINVESTAV UNIDAD TAMAULIPAS --- \today} % University, school and/or department name(s) and a date

%----------------------------------------------------------------------------------------

\begin{document}

\maketitle % Print the title

%----------------------------------------------------------------------------------------
%	INTRODUCTION
%----------------------------------------------------------------------------------------

Si durante su viaje a la posición final el robot detectara cambios en el mapa (un obstáculo en el camino o un nuevo paso abierto), requerirá volver a utilizar el algoritmo $A^{*}$ desde la posición actual. Una variante interesante del algoritmo $A^{*}$ es el algoritmo $D^{*}$, que permite reutilizar los cálculos realizados por el algoritmo $A^{*}$ en la etapa anterior, recalculando únicamente aquellas celdas que son directamente afectadas por el cambio detectado en el medio ambiente.\\

\begin{itemize}
\item Explicar el funcionamiento del algoritmo $D^{*}$.

  El algoritmo D* (D-star) es un algoritmo de búsqueda y planificación de trayectorias utilizado en robótica móvil para encontrar rutas óptimas y actualizadas en entornos con cambios dinámicos. A diferencia de otros algoritmos de búsqueda, como A* (A-star), el algoritmo D* permite adaptar y modificar la ruta planificada en tiempo real en respuesta a cambios en el entorno o en las condiciones del robot.\\

  El algoritmo D* se basa en el concepto de \textbf{grafo de búsqueda en sentido inverso}. En lugar de comenzar desde el punto de partida y expandirse hacia el objetivo como en A*, el algoritmo D* comienza desde el punto objetivo y se expande hacia atrás, actualizando los costos y las heurísticas en cada iteración.\\

  Descripción funcionamiento básico del algoritmo D*

  \begin{enumerate}
  \item Inicialización:
  \item Bucle principal:
  \item Actualización de celdas:
  \item Generación de trayectorias:
  \end{enumerate}
  
\item Documentar un artículo científico que emplee el algoritmo $D^{*}$ para la planificación de trayectorias.

  \textbf{Complete Coverage D* Algorithm for Path Planning of a Floor-Cleaning Mobile Robot} 
  
\item ¿Cuáles son las considereaciones del algoritmo?
  
  El algoritmo D* tiene la ventaja de poder reaccionar eficientemente a cambios en el entorno, ya que solo recalcula y actualiza las partes relevantes de la ruta planificada. Sin embargo, el algoritmo D* puede ser computacionalmente más costoso que otros algoritmos de búsqueda, ya que puede requerir más actualizaciones y propagaciones de costos.

  \begin{itemize}
  \item Representación del entorno: El algoritmo D* requiere una representación adecuada del entorno en el que opera el robot. Esto puede incluir un mapa o una retícula de celdas que indique la ocupación de cada posición en el espacio de trabajo. Es importante contar con una representación precisa y actualizada del entorno para lograr resultados precisos.
  \item Eficiencia computacional: Aunque el algoritmo D* ofrece la ventaja de adaptarse a cambios en tiempo real, también puede ser computacionalmente costoso. Las actualizaciones y propagaciones de costos pueden requerir un procesamiento adicional, por lo que es importante optimizar el algoritmo y buscar formas de reducir la carga computacional, especialmente en entornos de gran escala o con cambios frecuentes.
    
  \item Validación de la solución: Como en cualquier algoritmo de planificación de trayectorias, es fundamental verificar y validar la solución generada por el algoritmo D*. Esto implica comprobar que la trayectoria planificada cumple con los requisitos y restricciones del problema, como evitar colisiones con obstáculos, cumplir con restricciones de movimiento del robot y alcanzar el objetivo deseado.
  \end{itemize}
  
\end{itemize}

\end{document}
