%%%%%%%%%%%%%%%%%%%%%%%%%%%%%%%%%%%%%%%%%
% Lachaise Assignment
% LaTeX Template
% Version 1.0 (26/6/2018)
%
% This template originates from:
% http://www.LaTeXTemplates.com
%
% Authors:
% Marion Lachaise & François Févotte
% Vel (vel@LaTeXTemplates.com)
%
% License:
% CC BY-NC-SA 3.0 (http://creativecommons.org/licenses/by-nc-sa/3.0/)
% 
%%%%%%%%%%%%%%%%%%%%%%%%%%%%%%%%%%%%%%%%%

%----------------------------------------------------------------------------------------
%	PACKAGES AND OTHER DOCUMENT CONFIGURATIONS
%----------------------------------------------------------------------------------------

\documentclass{article}

\input{structure.tex} % Include the file specifying the document structure and custom commands

%----------------------------------------------------------------------------------------
%	ASSIGNMENT INFORMATION
%----------------------------------------------------------------------------------------

\title{TSRMI: Assignment \#13} % Title of the assignment

\author{Luis Alberto Ballado Aradias\\ \texttt{luis.ballado@cinvestav.mx}} % Author name and email address

\date{CINVESTAV UNIDAD TAMAULIPAS --- \today} % University, school and/or department name(s) and a date

%----------------------------------------------------------------------------------------

\begin{document}

\maketitle % Print the title

%----------------------------------------------------------------------------------------
%	INTRODUCTION
%----------------------------------------------------------------------------------------

Si durante su viaje a la posición final el robot detectara cambios en el mapa (un obstáculo en el camino o un nuevo paso abierto), requerirá volver a utilizar el algoritmo $A^{*}$ desde la posición actual. Una variante interesante del algoritmo $A^{*}$ es el algoritmo $D^{*}$, que permite reutilizar los cálculos realizados por el algoritmo $A^{*}$ en la etapa anterior, recalculando únicamente aquellas celdas que son directamente afectadas por el cambio detectado en el medio ambiente.\\

\begin{itemize}
\item Explicar el funcionamiento del algoritmo $D^{*}$.
\item Documentar un artículo científico que emplee el algoritmo $D^{*}$ para la planificación de trayectorias.
\item ¿Cuáles son las considereaciones del algoritmo?
\end{itemize}

\end{document}
