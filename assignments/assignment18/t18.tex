%%%%%%%%%%%%%%%%%%%%%%%%%%%%%%%%%%%%%%%%%
% Lachaise Assignment
% LaTeX Template
% Version 1.0 (26/6/2018)
%
% This template originates from:
% http://www.LaTeXTemplates.com
%
% Authors:
% Marion Lachaise & François Févotte
% Vel (vel@LaTeXTemplates.com)
%
% License:
% CC BY-NC-SA 3.0 (http://creativecommons.org/licenses/by-nc-sa/3.0/)
% 
%%%%%%%%%%%%%%%%%%%%%%%%%%%%%%%%%%%%%%%%%

%----------------------------------------------------------------------------------------
%	PACKAGES AND OTHER DOCUMENT CONFIGURATIONS
%----------------------------------------------------------------------------------------

\documentclass{article}

\input{structure.tex} % Include the file specifying the document structure and custom commands

%----------------------------------------------------------------------------------------
%	ASSIGNMENT INFORMATION
%----------------------------------------------------------------------------------------

\title{TSRMI: Assignment \#18} % Title of the assignment

\author{Luis Alberto Ballado Aradias\\ \texttt{luis.ballado@cinvestav.mx}} % Author name and email address

\date{CINVESTAV UNIDAD TAMAULIPAS --- \today} % University, school and/or department name(s) and a date

%----------------------------------------------------------------------------------------

\begin{document}

\maketitle % Print the title

%----------------------------------------------------------------------------------------
%	INTRODUCTION
%----------------------------------------------------------------------------------------

Documentar dos artículos científicos que realicen planificación de trayectorias en dos etapas: global y local.\\

\textbf{A Two-Stage Path Planning Method for Mobile Robots in Dynamic Environments}\\

En este artículo, se propone un método de planificación de trayectorias en dos etapas para robots móviles en entornos dinámicos. La primera etapa consiste en una planificación global que busca encontrar una trayectoria de alto nivel hacia el objetivo utilizando un algoritmo de búsqueda como el A* o Dijkstra. La segunda etapa se centra en la planificación local y tiene como objetivo refinar la trayectoria global considerando las condiciones y cambios en tiempo real del entorno. Se utiliza un algoritmo de navegación local, como el Campo Potencial o el RRT*, para ajustar la trayectoria global evitando obstáculos en movimiento y optimizando la eficiencia y seguridad de la navegación. El método propuesto se evalúa mediante simulaciones y pruebas en un entorno real, demostrando su eficacia para planificar trayectorias en entornos dinámicos.\\\\

\textbf{Hybrid Global-Local Path Planning for Autonomous Mobile Robots}

En este artículo, se presenta un enfoque híbrido de planificación de trayectorias que combina una etapa global y una etapa local para robots móviles autónomos. La etapa global se encarga de la planificación de alto nivel, utilizando técnicas como el algoritmo D* Lite o RRT-Connect, para encontrar una ruta óptima desde la posición inicial hasta el objetivo, teniendo en cuenta obstáculos estáticos. Una vez obtenida la ruta global, se pasa a la etapa local, donde se utiliza una técnica de planificación de trayectorias local, como el RRT*, para ajustar la trayectoria global considerando obstáculos dinámicos y evitando colisiones en tiempo real. El método se evalúa mediante simulaciones y pruebas en entornos reales, demostrando su capacidad para realizar una planificación de trayectorias eficiente y segura en presencia de obstáculos dinámicos.\\

Ambos enfoques son eficientes para manejar entornos dinámicos y han sido evaluados en simulaciones y entornos reales. La planificación global permite encontrar una ruta inicial y la planificación local ajusta esa ruta para adaptarse a las condiciones cambiantes del entorno, evitando obstáculos y optimizando la eficiencia de la navegación.

\begin{itemize}
\item ¿Qué tipo de arquitectura de control emplean?
  En ambos casos, la arquitectura de control basada en capas permite separar la planificación global de la planificación local, lo que facilita la modularidad y flexibilidad del sistema de control del robot. Esto permite un enfoque más eficiente y adaptable para la planificación de trayectorias en entornos dinámicos, ya que se pueden realizar ajustes locales en función de las condiciones cambiantes del entorno sin afectar la planificación global de alto nivel.
\item ¿Cuáles son los algoritmos que utilizan?
  \begin{enumerate}
  \item A Two-Stage Path Planning Method for Mobile Robots in Dynamic Environments, se utilizan los siguientes algoritmos:
    \begin{itemize}
    \item Algoritmo de búsqueda global: En la etapa de planificación global, se puede emplear un algoritmo de búsqueda como A* (A estrella) o Dijkstra para encontrar la ruta óptima desde la posición inicial hasta el objetivo. Estos algoritmos utilizan heurísticas y costos de movimiento para determinar la mejor ruta posible.
    \item Algoritmo de navegación local: En la etapa de planificación local, se utiliza un algoritmo de navegación local como el potencial de campos, que permite refinar la trayectoria global evitando obstáculos y optimizando el movimiento en tiempo real. Este algoritmo utiliza un campo de fuerza para guiar al robot hacia el objetivo y alejarlo de los obstáculos.
    \end{itemize}
  \item Hybrid Global-Local Path Planning for Autonomous Mobile Robots, se emplean los siguientes algoritmos
    \begin{itemize}
    \item Algoritmo de búsqueda global: En la etapa de planificación global, se puede utilizar un algoritmo de búsqueda como A* o Dijkstra para encontrar la ruta óptima desde la posición inicial hasta el objetivo, teniendo en cuenta obstáculos estáticos en el entorno.
    \item Algoritmo de planificación local: En la etapa de planificación local, se utiliza un algoritmo de planificación de trayectorias local como el RRT (Rapidly-Exploring Random Tree) para ajustar la ruta global considerando obstáculos dinámicos y evitando colisiones en tiempo real. El RRT genera una estructura de árbol aleatoria para explorar el espacio de configuración y encontrar una trayectoria viable.
    \end{itemize}
    
  \end{enumerate}

  En ambos casos, se combinan algoritmos de búsqueda global y planificación local para lograr una planificación de trayectorias eficiente y adaptable en entornos dinámicos. Los algoritmos utilizados se seleccionan según las necesidades específicas del problema y las características del entorno en el que opera el robot.
  
\item ¿De qué manera incluyen las restricciones mecánicas del vehículo dentro de la planificación?

  En la planificación de trayectorias, es importante tener en cuenta las restricciones mecánicas del vehículo, como su velocidad máxima, radio de giro mínimo, aceleración máxima, entre otros.

  \begin{enumerate}
  \item A Two-Stage Path Planning Method for Mobile Robots in Dynamic Environments, se menciona que durante la planificación global se tienen en cuenta las restricciones del vehículo al definir los costos de movimiento en el grafo de búsqueda. Esto implica asignar penalizaciones o costos más altos a los movimientos que violen las restricciones del vehículo, como giros bruscos o movimientos que excedan la velocidad máxima permitida. De esta manera, el algoritmo de búsqueda global seleccionará rutas que cumplan con las restricciones mecánicas del vehículo.

  \item Hybrid Global-Local Path Planning for Autonomous Mobile Robots, se menciona que en la etapa de planificación local se emplea un algoritmo de navegación local basado en el potencial de campos. Este enfoque permite modelar y tener en cuenta las restricciones mecánicas del vehículo al generar el campo de fuerza. Por ejemplo, se pueden establecer zonas de atracción y repulsión que reflejen la velocidad y el radio de giro del vehículo, de modo que se eviten movimientos que violen dichas restricciones. Además, el algoritmo de navegación local puede considerar otras restricciones, como evitar colisiones con obstáculos o mantener una distancia de seguridad.
    
  \end{enumerate}
  En resumen, ambos artículos abordan las restricciones mecánicas del vehículo al diseñar los algoritmos de planificación de trayectorias. Ya sea mediante la asignación de costos en la planificación global o mediante la generación de campos de fuerza en la planificación local, se busca garantizar que las trayectorias generadas sean factibles y cumplan con las limitaciones físicas del vehículo.
\end{itemize}

\end{document}
