%%%%%%%%%%%%%%%%%%%%%%%%%%%%%%%%%%%%%%%%%
% Lachaise Assignment
% LaTeX Template
% Version 1.0 (26/6/2018)
%
% This template originates from:
% http://www.LaTeXTemplates.com
%
% Authors:
% Marion Lachaise & François Févotte
% Vel (vel@LaTeXTemplates.com)
%
% License:
% CC BY-NC-SA 3.0 (http://creativecommons.org/licenses/by-nc-sa/3.0/)
% 
%%%%%%%%%%%%%%%%%%%%%%%%%%%%%%%%%%%%%%%%%

%----------------------------------------------------------------------------------------
%	PACKAGES AND OTHER DOCUMENT CONFIGURATIONS
%----------------------------------------------------------------------------------------

\documentclass{article}

\input{structure.tex} % Include the file specifying the document structure and custom commands

%----------------------------------------------------------------------------------------
%	ASSIGNMENT INFORMATION
%----------------------------------------------------------------------------------------

\title{TSRMI: Assignment \#12} % Title of the assignment

\author{Luis Alberto Ballado Aradias\\ \texttt{luis.ballado@cinvestav.mx}} % Author name and email address

\date{CINVESTAV UNIDAD TAMAULIPAS --- \today} % University, school and/or department name(s) and a date

%----------------------------------------------------------------------------------------

\begin{document}

\maketitle % Print the title

%----------------------------------------------------------------------------------------
%	INTRODUCTION
%----------------------------------------------------------------------------------------

La implementación \textit{natural} de un algoritmo de frente de onda, empleando una pila para almacenar los nodos visitados, no es la más eficiente. Una forma más sencilla y eficiente es emplear una variante del algoritmo de \textbf{chamfer}, utilizado para calcular la transformada de distancia de una imagen binaria. El algoritmo de chamfer se basa en el barrido sucesivo de una(s) máscara(s) de distancias sobre la retícula de la imagen.\\

\begin{itemize}
\item Explicar el funcionamiento del algoritmo de chamfer.
\item Documentar un artículo científico que emplee el algoritmo de chamfer para la planificación de trayectorias (variante propuesta por M. A. Jarvis).
\item ¿Cuáles son las consideraciones del algoritmo?
\item ¿Cómo se modifica la máscara para calcular otras distancias?
\item ¿Cuál es la diferencia entre las trayectorias obtenidas cuando se emplean distancias distintas?
\end{itemize}

\end{document}
