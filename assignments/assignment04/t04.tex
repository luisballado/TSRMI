%%%%%%%%%%%%%%%%%%%%%%%%%%%%%%%%%%%%%%%%%
% Lachaise Assignment
% LaTeX Template
% Version 1.0 (26/6/2018)
%
% This template originates from:
% http://www.LaTeXTemplates.com
%
% Authors:
% Marion Lachaise & François Févotte
% Vel (vel@LaTeXTemplates.com)
%
% License:
% CC BY-NC-SA 3.0 (http://creativecommons.org/licenses/by-nc-sa/3.0/)
% 
%%%%%%%%%%%%%%%%%%%%%%%%%%%%%%%%%%%%%%%%%

%----------------------------------------------------------------------------------------
%	PACKAGES AND OTHER DOCUMENT CONFIGURATIONS
%----------------------------------------------------------------------------------------

\documentclass{article}

%%%%%%%%%%%%%%%%%%%%%%%%%%%%%%%%%%%%%%%%%
% Lachaise Assignment
% Structure Specification File
% Version 1.0 (26/6/2018)
%
% This template originates from:
% http://www.LaTeXTemplates.com
%
% Authors:
% Marion Lachaise & François Févotte
% Vel (vel@LaTeXTemplates.com)
%
% License:
% CC BY-NC-SA 3.0 (http://creativecommons.org/licenses/by-nc-sa/3.0/)
% 
%%%%%%%%%%%%%%%%%%%%%%%%%%%%%%%%%%%%%%%%%

%----------------------------------------------------------------------------------------
%	PACKAGES AND OTHER DOCUMENT CONFIGURATIONS
%----------------------------------------------------------------------------------------

\usepackage{amsmath,amsfonts,stmaryrd,amssymb} % Math packages

\usepackage{enumerate} % Custom item numbers for enumerations
\usepackage{longtable} % To display tables on several pages
\usepackage{rotating}
\usepackage[ruled]{algorithm2e} % Algorithms
\usepackage[spanish]{babel}
\usepackage[framemethod=tikz]{mdframed} % Allows defining custom boxed/framed environments

\usepackage{listings} % File listings, with syntax highlighting
\lstset{
	basicstyle=\ttfamily, % Typeset listings in monospace font
}

%----------------------------------------------------------------------------------------
%	DOCUMENT MARGINS
%----------------------------------------------------------------------------------------

\usepackage{geometry} % Required for adjusting page dimensions and margins

\geometry{
	paper=a4paper, % Paper size, change to letterpaper for US letter size
	top=2.5cm, % Top margin
	bottom=3cm, % Bottom margin
	left=2.5cm, % Left margin
	right=2.5cm, % Right margin
	headheight=14pt, % Header height
	footskip=1.5cm, % Space from the bottom margin to the baseline of the footer
	headsep=1.2cm, % Space from the top margin to the baseline of the header
	%showframe, % Uncomment to show how the type block is set on the page
}

%----------------------------------------------------------------------------------------
%	FONTS
%----------------------------------------------------------------------------------------

\usepackage[utf8]{inputenc} % Required for inputting international characters
\usepackage[T1]{fontenc} % Output font encoding for international characters

\usepackage{XCharter} % Use the XCharter fonts

%----------------------------------------------------------------------------------------
%	COMMAND LINE ENVIRONMENT
%----------------------------------------------------------------------------------------

% Usage:
% \begin{commandline}
%	\begin{verbatim}
%		$ ls
%		
%		Applications	Desktop	...
%	\end{verbatim}
% \end{commandline}

\mdfdefinestyle{commandline}{
	leftmargin=10pt,
	rightmargin=10pt,
	innerleftmargin=15pt,
	middlelinecolor=black!50!white,
	middlelinewidth=2pt,
	frametitlerule=false,
	backgroundcolor=black!5!white,
	frametitle={Command Line},
	frametitlefont={\normalfont\sffamily\color{white}\hspace{-1em}},
	frametitlebackgroundcolor=black!50!white,
	nobreak,
}

% Define a custom environment for command-line snapshots
\newenvironment{commandline}{
	\medskip
	\begin{mdframed}[style=commandline]
}{
	\end{mdframed}
	\medskip
}

%----------------------------------------------------------------------------------------
%	FILE CONTENTS ENVIRONMENT
%----------------------------------------------------------------------------------------

% Usage:
% \begin{file}[optional filename, defaults to "File"]
%	File contents, for example, with a listings environment
% \end{file}

\mdfdefinestyle{file}{
	innertopmargin=1.6\baselineskip,
	innerbottommargin=0.8\baselineskip,
	topline=false, bottomline=false,
	leftline=false, rightline=false,
	leftmargin=2cm,
	rightmargin=2cm,
	singleextra={%
		\draw[fill=black!10!white](P)++(0,-1.2em)rectangle(P-|O);
		\node[anchor=north west]
		at(P-|O){\ttfamily\mdfilename};
		%
		\def\l{3em}
		\draw(O-|P)++(-\l,0)--++(\l,\l)--(P)--(P-|O)--(O)--cycle;
		\draw(O-|P)++(-\l,0)--++(0,\l)--++(\l,0);
	},
	nobreak,
}

% Define a custom environment for file contents
\newenvironment{file}[1][File]{ % Set the default filename to "File"
	\medskip
	\newcommand{\mdfilename}{#1}
	\begin{mdframed}[style=file]
}{
	\end{mdframed}
	\medskip
}

%----------------------------------------------------------------------------------------
%	NUMBERED QUESTIONS ENVIRONMENT
%----------------------------------------------------------------------------------------

% Usage:
% \begin{question}[optional title]
%	Question contents
% \end{question}

\mdfdefinestyle{question}{
	innertopmargin=1.2\baselineskip,
	innerbottommargin=0.8\baselineskip,
	roundcorner=5pt,
	nobreak,
	singleextra={%
		\draw(P-|O)node[xshift=1em,anchor=west,fill=white,draw,rounded corners=5pt]{%
		Question \theQuestion\questionTitle};
	},
}

\newcounter{Question} % Stores the current question number that gets iterated with each new question

% Define a custom environment for numbered questions
\newenvironment{question}[1][\unskip]{
	\bigskip
	\stepcounter{Question}
	\newcommand{\questionTitle}{~#1}
	\begin{mdframed}[style=question]
}{
	\end{mdframed}
	\medskip
}

%----------------------------------------------------------------------------------------
%	WARNING TEXT ENVIRONMENT
%----------------------------------------------------------------------------------------

% Usage:
% \begin{warn}[optional title, defaults to "Warning:"]
%	Contents
% \end{warn}

\mdfdefinestyle{warning}{
	topline=false, bottomline=false,
	leftline=false, rightline=false,
	nobreak,
	singleextra={%
		\draw(P-|O)++(-0.5em,0)node(tmp1){};
		\draw(P-|O)++(0.5em,0)node(tmp2){};
		\fill[black,rotate around={45:(P-|O)}](tmp1)rectangle(tmp2);
		\node at(P-|O){\color{white}\scriptsize\bf !};
		\draw[very thick](P-|O)++(0,-1em)--(O);%--(O-|P);
	}
}

% Define a custom environment for warning text
\newenvironment{warn}[1][Warning:]{ % Set the default warning to "Warning:"
	\medskip
	\begin{mdframed}[style=warning]
		\noindent{\textbf{#1}}
}{
	\end{mdframed}
}

%----------------------------------------------------------------------------------------
%	INFORMATION ENVIRONMENT
%----------------------------------------------------------------------------------------

% Usage:
% \begin{info}[optional title, defaults to "Info:"]
% 	contents
% 	\end{info}

\mdfdefinestyle{info}{%
	topline=false, bottomline=false,
	leftline=false, rightline=false,
	nobreak,
	singleextra={%
		\fill[black](P-|O)circle[radius=0.4em];
		\node at(P-|O){\color{white}\scriptsize\bf i};
		\draw[very thick](P-|O)++(0,-0.8em)--(O);%--(O-|P);
	}
}

% Define a custom environment for information
\newenvironment{info}[1][Info:]{ % Set the default title to "Info:"
	\medskip
	\begin{mdframed}[style=info]
		\noindent{\textbf{#1}}
}{
	\end{mdframed}
}
 % Include the file specifying the document structure and custom commands

%----------------------------------------------------------------------------------------
%	ASSIGNMENT INFORMATION
%----------------------------------------------------------------------------------------

\title{TSRMI: Assignment \#4} % Title of the assignment

\author{Luis Alberto Ballado Aradias\\ \texttt{luis.ballado@cinvestav.mx}} % Author name and email address

\date{CINVESTAV UNIDAD TAMAULIPAS --- \today} % University, school and/or department name(s) and a date

%----------------------------------------------------------------------------------------

\begin{document}

\maketitle % Print the title

%----------------------------------------------------------------------------------------
%	INTRODUCTION
%----------------------------------------------------------------------------------------

\begin{itemize}
\item Documentar las distintas maneras en que se puede implementar un polígono en la memoria de una computadora.

  Existen varias formas para implementar un poligono en la memoria de una computadora.\\

  \begin{enumerate}
  \item \textbf{Lista de coordenadas:} En esta implementación, se utiliza una lista de coordenadas para representar los vértices del polígono. Cada vértice se define por sus coordenadas x e y. La secuencia de vértices en la lista determina el orden en el que se conectan para formar los lados del polígono.
  \item \textbf{Lista de puntos:} Similar a la implementación anterior, se utiliza una lista de puntos para representar los vértices del polígono. Cada punto contiene una estructura de datos con campos para las coordenadas x e y. La lista de puntos se organiza en el orden deseado para definir los lados del polígono.
  \item \textbf{Arreglo de coordenadas:} En esta implementación, se utiliza un arreglo unidimensional para almacenar las coordenadas x e y de los vértices del polígono. Cada par de coordenadas consecutivas forma un vértice. La longitud del arreglo determina el número de vértices del polígono.
  \item \textbf{Estructura de datos poligono:} Se puede definir una estructura de datos personalizada para representar un polígono. Esta estructura de datos puede contener información adicional, como el número de vértices, el tipo de polígono (regular, irregular, convexo, cóncavo, etc.), propiedades geométricas adicionales, etc.
  \item \textbf{Representación matemática:} En lugar de almacenar explícitamente los vértices del polígono, se puede utilizar una representación matemática, como una ecuación paramétrica o una ecuación de curva, para definir la forma del polígono. Esta representación se puede utilizar para realizar cálculos y operaciones geométricas, como intersecciones, rotaciones, transformaciones, etc. 
  \end{enumerate}

  La elección de la forma de implementación dependerá de las necesidades y características específicas del sistema o aplicación en la que se esté trabajando. Cada enfoque tiene sus ventajas y desventajas en términos de eficiencia de almacenamiento, facilidad de manipulación, capacidad de realizar operaciones geométricas, entre otros factores. Es importante considerar también el lenguaje de programación utilizado y las bibliotecas disponibles para el manejo de geometría y gráficos.
  
\item Documentar las distintas maneras en que se puede implementar una línea infinita en la memoria de una computadora.

  La representación de una línea infinita en la memoria de una computadora es un problema interesante debido a su naturaleza continua y no acotada. A continuación, se presentan algunas técnicas comunes utilizadas para implementar una línea infinita en la memoria de una computadora:

  \begin{enumerate}
  \item \textbf{Ecuación de la línea:} Una forma común de representar una línea infinita es mediante su ecuación matemática. La ecuación de la línea generalmente se expresa en la forma y = mx + b, donde m es la pendiente de la línea y b es el término de desplazamiento. Esta representación permite almacenar los coeficientes m y b en variables en la memoria de la computadora.
    
  \item \textbf{Coordenadas extremas:} Otra forma de representar una línea infinita es mediante el almacenamiento de las coordenadas de dos puntos extremos en la línea. Estos puntos definen la dirección y la pendiente de la línea. Sin embargo, es importante tener en cuenta que esta representación solo es adecuada cuando se necesita almacenar segmentos finitos de la línea, en lugar de la línea completa.

  \item \textbf{Vector de dirección:} En esta implementación, se utiliza un vector para representar la dirección y pendiente de la línea. El vector contiene componentes para la magnitud y la dirección, y se puede almacenar en la memoria de la computadora.

  \item \textbf{Ecuación paramétrica:} En lugar de utilizar una ecuación explícita, se puede utilizar una ecuación paramétrica para representar una línea infinita. Esta representación utiliza parámetros para generar puntos en la línea y puede ser útil para realizar cálculos y operaciones geométricas.

  \item \textbf{Puntos de muestreo:} En algunos casos, se puede implementar una línea infinita mediante la generación de un conjunto discreto de puntos de muestreo a lo largo de la línea. Estos puntos se almacenan en una estructura de datos, como una lista o un arreglo, y se utilizan para aproximar la línea continua.
    
  \end{enumerate}
  
  
\item{Encontrar desarrollos científicos en robótica móvil que empleen este tipo de representaciones para sus mapas del medio ambiente}

  \begin{enumerate}
  \item \textbf{SLAM basado en líneas infinitas}: En el artículo \textbf{Infinitam: An efficient method for SLAM using lines}, los investigadores desarrollaron un método eficiente para la construcción de mapas del entorno utilizando líneas infinitas. En lugar de representar el entorno con mallas o puntos discretos, utilizaron líneas infinitas para capturar la estructura y la geometría del entorno de manera más precisa. Esto permitió una representación más compacta y una mejor estimación del movimiento del robot. El algoritmo utilizado combinaba información visual y geométrica para construir y actualizar el mapa del entorno.
  \item \textbf{Navegación basada en lineas infinitas}: En el artículo \textbf{Line-Based Path Planning for Mobile Robots}, los investigadores propusieron un enfoque de planificación de trayectorias para robots móviles basado en líneas infinitas. Utilizaron líneas infinitas como características del entorno para generar rutas seguras y eficientes para el robot. El mapa del entorno se construyó utilizando líneas detectadas por sensores, como cámaras o láseres. Luego, se aplicaron técnicas de búsqueda y optimización para encontrar la trayectoria óptima basada en la información de las líneas infinitas.
  \end{enumerate}

  Estos desarrollos demuestran cómo las representaciones de líneas infinitas pueden ser utilizadas de manera efectiva en la robótica móvil para construir mapas del medio ambiente y realizar tareas como la localización, la navegación y la planificación de trayectorias. Al utilizar líneas infinitas en lugar de representaciones discretas, se puede capturar mejor la estructura y la geometría del entorno, lo que permite una mayor precisión y eficiencia en las operaciones del robot.

  \item {¿Qué otras primitivas geométricas podrían utilizarse para construir un mapa del medio ambiente?}

    
    \begin{itemize}
    \item \textbf{Circulos:} Los círculos son primitivas geométricas utilizadas para representar objetos redondos o áreas circulares en el entorno. Se utilizan comúnmente para representar obstáculos circulares, áreas de cobertura o regiones de interés. 
    \item \textbf{Elipses:} Las elipses son primitivas similares a los círculos, pero tienen una forma ovalada. Se utilizan para representar objetos o áreas con formas elípticas en el entorno.
    \item \textbf{Rectángulos:} Los rectángulos son polígonos con cuatro ángulos rectos. Se utilizan para representar objetos rectangulares, como mesas, cajas o puertas, así como áreas o regiones rectangulares en el entorno. 
    \end{itemize}
\end{itemize}

\end{document}
