%%%%%%%%%%%%%%%%%%%%%%%%%%%%%%%%%%%%%%%%%
% Lachaise Assignment
% LaTeX Template
% Version 1.0 (26/6/2018)
%
% This template originates from:
% http://www.LaTeXTemplates.com
%
% Authors:
% Marion Lachaise & François Févotte
% Vel (vel@LaTeXTemplates.com)
%
% License:
% CC BY-NC-SA 3.0 (http://creativecommons.org/licenses/by-nc-sa/3.0/)
% 
%%%%%%%%%%%%%%%%%%%%%%%%%%%%%%%%%%%%%%%%%

%----------------------------------------------------------------------------------------
%	PACKAGES AND OTHER DOCUMENT CONFIGURATIONS
%----------------------------------------------------------------------------------------

\documentclass{article}

\input{structure.tex} % Include the file specifying the document structure and custom commands

%----------------------------------------------------------------------------------------
%	ASSIGNMENT INFORMATION
%----------------------------------------------------------------------------------------

\title{TSRMI: Assignment \#4} % Title of the assignment

\author{Luis Alberto Ballado Aradias\\ \texttt{luis.ballado@cinvestav.mx}} % Author name and email address

\date{CINVESTAV UNIDAD TAMAULIPAS --- \today} % University, school and/or department name(s) and a date

%----------------------------------------------------------------------------------------

\begin{document}

\maketitle % Print the title

%----------------------------------------------------------------------------------------
%	INTRODUCTION
%----------------------------------------------------------------------------------------

\section*{TAREA} % Unnumbered section

\begin{itemize}
\item Documentar las distintas maneras en que se puede implementar un polígono en la memoria de una computadora.
\item Documentar las distintas maneras en que se puede implementar una línea infinita en la memoria de una computadora.
\end{itemize}

\newpage
\section{Encontrar desarrollos científicos en robótica móvil que empleen este tipo de representaciones para sus mapas del medio ambiente} % Numbered section


\begin{figure}[h]
\includegraphics[width=10cm]{images/vant.jpg}
\centering
\end{figure}

%------------------------------------------------
\newpage
\section{¿Qué otras primitivas geométricas podrían utilizarse para construir un mapa del medio ambiente?}

\begin{figure}[h]
\includegraphics[width=10cm]{images/drone_alafija.jpg}
\centering
\end{figure}

\end{document}
