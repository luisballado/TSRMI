%%%%%%%%%%%%%%%%%%%%%%%%%%%%%%%%%%%%%%%%%
% Lachaise Assignment
% LaTeX Template
% Version 1.0 (26/6/2018)
%
% This template originates from:
% http://www.LaTeXTemplates.com
%
% Authors:
% Marion Lachaise & François Févotte
% Vel (vel@LaTeXTemplates.com)
%
% License:
% CC BY-NC-SA 3.0 (http://creativecommons.org/licenses/by-nc-sa/3.0/)
% 
%%%%%%%%%%%%%%%%%%%%%%%%%%%%%%%%%%%%%%%%%

%----------------------------------------------------------------------------------------
%	PACKAGES AND OTHER DOCUMENT CONFIGURATIONS
%----------------------------------------------------------------------------------------

\documentclass{article}

%%%%%%%%%%%%%%%%%%%%%%%%%%%%%%%%%%%%%%%%%
% Lachaise Assignment
% Structure Specification File
% Version 1.0 (26/6/2018)
%
% This template originates from:
% http://www.LaTeXTemplates.com
%
% Authors:
% Marion Lachaise & François Févotte
% Vel (vel@LaTeXTemplates.com)
%
% License:
% CC BY-NC-SA 3.0 (http://creativecommons.org/licenses/by-nc-sa/3.0/)
% 
%%%%%%%%%%%%%%%%%%%%%%%%%%%%%%%%%%%%%%%%%

%----------------------------------------------------------------------------------------
%	PACKAGES AND OTHER DOCUMENT CONFIGURATIONS
%----------------------------------------------------------------------------------------

\usepackage{amsmath,amsfonts,stmaryrd,amssymb} % Math packages

\usepackage{enumerate} % Custom item numbers for enumerations
\usepackage{longtable} % To display tables on several pages
\usepackage{rotating}
\usepackage[ruled]{algorithm2e} % Algorithms
\usepackage[spanish]{babel}
\usepackage[framemethod=tikz]{mdframed} % Allows defining custom boxed/framed environments

\usepackage{listings} % File listings, with syntax highlighting
\lstset{
	basicstyle=\ttfamily, % Typeset listings in monospace font
}

%----------------------------------------------------------------------------------------
%	DOCUMENT MARGINS
%----------------------------------------------------------------------------------------

\usepackage{geometry} % Required for adjusting page dimensions and margins

\geometry{
	paper=a4paper, % Paper size, change to letterpaper for US letter size
	top=2.5cm, % Top margin
	bottom=3cm, % Bottom margin
	left=2.5cm, % Left margin
	right=2.5cm, % Right margin
	headheight=14pt, % Header height
	footskip=1.5cm, % Space from the bottom margin to the baseline of the footer
	headsep=1.2cm, % Space from the top margin to the baseline of the header
	%showframe, % Uncomment to show how the type block is set on the page
}

%----------------------------------------------------------------------------------------
%	FONTS
%----------------------------------------------------------------------------------------

\usepackage[utf8]{inputenc} % Required for inputting international characters
\usepackage[T1]{fontenc} % Output font encoding for international characters

\usepackage{XCharter} % Use the XCharter fonts

%----------------------------------------------------------------------------------------
%	COMMAND LINE ENVIRONMENT
%----------------------------------------------------------------------------------------

% Usage:
% \begin{commandline}
%	\begin{verbatim}
%		$ ls
%		
%		Applications	Desktop	...
%	\end{verbatim}
% \end{commandline}

\mdfdefinestyle{commandline}{
	leftmargin=10pt,
	rightmargin=10pt,
	innerleftmargin=15pt,
	middlelinecolor=black!50!white,
	middlelinewidth=2pt,
	frametitlerule=false,
	backgroundcolor=black!5!white,
	frametitle={Command Line},
	frametitlefont={\normalfont\sffamily\color{white}\hspace{-1em}},
	frametitlebackgroundcolor=black!50!white,
	nobreak,
}

% Define a custom environment for command-line snapshots
\newenvironment{commandline}{
	\medskip
	\begin{mdframed}[style=commandline]
}{
	\end{mdframed}
	\medskip
}

%----------------------------------------------------------------------------------------
%	FILE CONTENTS ENVIRONMENT
%----------------------------------------------------------------------------------------

% Usage:
% \begin{file}[optional filename, defaults to "File"]
%	File contents, for example, with a listings environment
% \end{file}

\mdfdefinestyle{file}{
	innertopmargin=1.6\baselineskip,
	innerbottommargin=0.8\baselineskip,
	topline=false, bottomline=false,
	leftline=false, rightline=false,
	leftmargin=2cm,
	rightmargin=2cm,
	singleextra={%
		\draw[fill=black!10!white](P)++(0,-1.2em)rectangle(P-|O);
		\node[anchor=north west]
		at(P-|O){\ttfamily\mdfilename};
		%
		\def\l{3em}
		\draw(O-|P)++(-\l,0)--++(\l,\l)--(P)--(P-|O)--(O)--cycle;
		\draw(O-|P)++(-\l,0)--++(0,\l)--++(\l,0);
	},
	nobreak,
}

% Define a custom environment for file contents
\newenvironment{file}[1][File]{ % Set the default filename to "File"
	\medskip
	\newcommand{\mdfilename}{#1}
	\begin{mdframed}[style=file]
}{
	\end{mdframed}
	\medskip
}

%----------------------------------------------------------------------------------------
%	NUMBERED QUESTIONS ENVIRONMENT
%----------------------------------------------------------------------------------------

% Usage:
% \begin{question}[optional title]
%	Question contents
% \end{question}

\mdfdefinestyle{question}{
	innertopmargin=1.2\baselineskip,
	innerbottommargin=0.8\baselineskip,
	roundcorner=5pt,
	nobreak,
	singleextra={%
		\draw(P-|O)node[xshift=1em,anchor=west,fill=white,draw,rounded corners=5pt]{%
		Question \theQuestion\questionTitle};
	},
}

\newcounter{Question} % Stores the current question number that gets iterated with each new question

% Define a custom environment for numbered questions
\newenvironment{question}[1][\unskip]{
	\bigskip
	\stepcounter{Question}
	\newcommand{\questionTitle}{~#1}
	\begin{mdframed}[style=question]
}{
	\end{mdframed}
	\medskip
}

%----------------------------------------------------------------------------------------
%	WARNING TEXT ENVIRONMENT
%----------------------------------------------------------------------------------------

% Usage:
% \begin{warn}[optional title, defaults to "Warning:"]
%	Contents
% \end{warn}

\mdfdefinestyle{warning}{
	topline=false, bottomline=false,
	leftline=false, rightline=false,
	nobreak,
	singleextra={%
		\draw(P-|O)++(-0.5em,0)node(tmp1){};
		\draw(P-|O)++(0.5em,0)node(tmp2){};
		\fill[black,rotate around={45:(P-|O)}](tmp1)rectangle(tmp2);
		\node at(P-|O){\color{white}\scriptsize\bf !};
		\draw[very thick](P-|O)++(0,-1em)--(O);%--(O-|P);
	}
}

% Define a custom environment for warning text
\newenvironment{warn}[1][Warning:]{ % Set the default warning to "Warning:"
	\medskip
	\begin{mdframed}[style=warning]
		\noindent{\textbf{#1}}
}{
	\end{mdframed}
}

%----------------------------------------------------------------------------------------
%	INFORMATION ENVIRONMENT
%----------------------------------------------------------------------------------------

% Usage:
% \begin{info}[optional title, defaults to "Info:"]
% 	contents
% 	\end{info}

\mdfdefinestyle{info}{%
	topline=false, bottomline=false,
	leftline=false, rightline=false,
	nobreak,
	singleextra={%
		\fill[black](P-|O)circle[radius=0.4em];
		\node at(P-|O){\color{white}\scriptsize\bf i};
		\draw[very thick](P-|O)++(0,-0.8em)--(O);%--(O-|P);
	}
}

% Define a custom environment for information
\newenvironment{info}[1][Info:]{ % Set the default title to "Info:"
	\medskip
	\begin{mdframed}[style=info]
		\noindent{\textbf{#1}}
}{
	\end{mdframed}
}
 % Include the file specifying the document structure and custom commands

%----------------------------------------------------------------------------------------
%	ASSIGNMENT INFORMATION
%----------------------------------------------------------------------------------------

\title{TSRMI: Assignment \#18} % Title of the assignment

\author{Luis Alberto Ballado Aradias\\ \texttt{luis.ballado@cinvestav.mx}} % Author name and email address

\date{CINVESTAV UNIDAD TAMAULIPAS --- \today} % University, school and/or department name(s) and a date

%----------------------------------------------------------------------------------------

\begin{document}

\maketitle % Print the title

%----------------------------------------------------------------------------------------
%	INTRODUCTION
%----------------------------------------------------------------------------------------

Documentar dos artículos científicos que realicen planificación de trayectorias en dos etapas: global y local.\\

\textbf{A Hybrid Path Planning Algorithm for Unmanned Surface Vehicles in Complex Environment With Dynamic Obstacles}\\

En este artículo, se propone un método de planificación de trayectorias en dos etapas para robots móviles en entornos dinámicos. La primera etapa consiste en una planificación global que busca encontrar una trayectoria de alto nivel hacia el objetivo utilizando un algoritmo de búsqueda como el A* o Dijkstra. La segunda etapa se centra en la planificación local y tiene como objetivo refinar la trayectoria global considerando las condiciones y cambios en tiempo real del entorno. Se utiliza un algoritmo de navegación local, como el Campo Potencial o el RRT*, para ajustar la trayectoria global evitando obstáculos en movimiento y optimizando la eficiencia y seguridad de la navegación. El método propuesto se evalúa mediante simulaciones y pruebas en un entorno real, demostrando su eficacia para planificar trayectorias en entornos dinámicos.\\\\

\textbf{Combining Global and Local Planning with Guarantees on Completeness}\\

En este artículo, se presenta un enfoque híbrido de planificación de trayectorias que combina una etapa global y una etapa local para robots móviles autónomos. La etapa global se encarga de la planificación de alto nivel, utilizando técnicas como el algoritmo D* Lite o RRT-Connect, para encontrar una ruta óptima desde la posición inicial hasta el objetivo, teniendo en cuenta obstáculos estáticos. Una vez obtenida la ruta global, se pasa a la etapa local, donde se utiliza una técnica de planificación de trayectorias local, como el RRT*, para ajustar la trayectoria global considerando obstáculos dinámicos y evitando colisiones en tiempo real. El método se evalúa mediante simulaciones y pruebas en entornos reales, demostrando su capacidad para realizar una planificación de trayectorias eficiente y segura en presencia de obstáculos dinámicos.\\

Ambos enfoques son eficientes para manejar entornos dinámicos y han sido evaluados en simulaciones y entornos reales. La planificación global permite encontrar una ruta inicial y la planificación local ajusta esa ruta para adaptarse a las condiciones cambiantes del entorno, evitando obstáculos y optimizando la eficiencia de la navegación.

\begin{itemize}
\item ¿Qué tipo de arquitectura de control emplean?
  En ambos casos, la arquitectura de control basada en capas permite separar la planificación global de la planificación local, lo que facilita la modularidad y flexibilidad del sistema de control del robot. Esto permite un enfoque más eficiente y adaptable para la planificación de trayectorias en entornos dinámicos, ya que se pueden realizar ajustes locales en función de las condiciones cambiantes del entorno sin afectar la planificación global de alto nivel.
\item ¿Cuáles son los algoritmos que utilizan?
  \begin{enumerate}
  \item \textbf{A Hybrid Path Planning Algorithm for Unmanned Surface Vehicles in Complex Environment With Dynamic Obstacles}, se utilizan los siguientes algoritmos:
    \begin{itemize}
    \item Algoritmo de búsqueda global: En la etapa de planificación global, se puede emplear un algoritmo de búsqueda como A* (A estrella) o Dijkstra para encontrar la ruta óptima desde la posición inicial hasta el objetivo. Estos algoritmos utilizan heurísticas y costos de movimiento para determinar la mejor ruta posible.
    \item Algoritmo de navegación local: En la etapa de planificación local, se utiliza un algoritmo de navegación local como el potencial de campos, que permite refinar la trayectoria global evitando obstáculos y optimizando el movimiento en tiempo real. Este algoritmo utiliza un campo de fuerza para guiar al robot hacia el objetivo y alejarlo de los obstáculos.
    \end{itemize}
  \item \textbf{Combining Global and Local Planning with Guarantees on Completeness}, se emplean los siguientes algoritmos
    \begin{itemize}
    \item Algoritmo de búsqueda global: En la etapa de planificación global, se puede utilizar un algoritmo de búsqueda como A* o Dijkstra para encontrar la ruta óptima desde la posición inicial hasta el objetivo, teniendo en cuenta obstáculos estáticos en el entorno.
    \item Algoritmo de planificación local: En la etapa de planificación local, se utiliza un algoritmo de planificación de trayectorias local como el RRT (Rapidly-Exploring Random Tree) para ajustar la ruta global considerando obstáculos dinámicos y evitando colisiones en tiempo real. El RRT genera una estructura de árbol aleatoria para explorar el espacio de configuración y encontrar una trayectoria viable.
    \end{itemize}
    
  \end{enumerate}

  En ambos casos, se combinan algoritmos de búsqueda global y planificación local para lograr una planificación de trayectorias eficiente y adaptable en entornos dinámicos. Los algoritmos utilizados se seleccionan según las necesidades específicas del problema y las características del entorno en el que opera el robot.
  
\item ¿De qué manera incluyen las restricciones mecánicas del vehículo dentro de la planificación?

  En la planificación de trayectorias, es importante tener en cuenta las restricciones mecánicas del vehículo, como su velocidad máxima, radio de giro mínimo, aceleración máxima, entre otros.

  \begin{enumerate}
  \item \textbf{A Hybrid Path Planning Algorithm for Unmanned Surface Vehicles in Complex Environment With Dynamic Obstacles}, se menciona que durante la planificación global se tienen en cuenta las restricciones del vehículo al definir los costos de movimiento en el grafo de búsqueda. Esto implica asignar penalizaciones o costos más altos a los movimientos que violen las restricciones del vehículo, como giros bruscos o movimientos que excedan la velocidad máxima permitida. De esta manera, el algoritmo de búsqueda global seleccionará rutas que cumplan con las restricciones mecánicas del vehículo.

  \item \textbf{Combining Global and Local Planning with Guarantees on Completeness}, se menciona que en la etapa de planificación local se emplea un algoritmo de navegación local basado en el potencial de campos. Este enfoque permite modelar y tener en cuenta las restricciones mecánicas del vehículo al generar el campo de fuerza. Por ejemplo, se pueden establecer zonas de atracción y repulsión que reflejen la velocidad y el radio de giro del vehículo, de modo que se eviten movimientos que violen dichas restricciones. Además, el algoritmo de navegación local puede considerar otras restricciones, como evitar colisiones con obstáculos o mantener una distancia de seguridad.
    
  \end{enumerate}
  
\end{itemize}

\end{document}
