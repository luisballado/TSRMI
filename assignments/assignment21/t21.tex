%%%%%%%%%%%%%%%%%%%%%%%%%%%%%%%%%%%%%%%%%
% Lachaise Assignment
% LaTeX Template
% Version 1.0 (26/6/2018)
%
% This template originates from:
% http://www.LaTeXTemplates.com
%
% Authors:
% Marion Lachaise & François Févotte
% Vel (vel@LaTeXTemplates.com)
%
% License:
% CC BY-NC-SA 3.0 (http://creativecommons.org/licenses/by-nc-sa/3.0/)
% 
%%%%%%%%%%%%%%%%%%%%%%%%%%%%%%%%%%%%%%%%%

%----------------------------------------------------------------------------------------
%	PACKAGES AND OTHER DOCUMENT CONFIGURATIONS
%----------------------------------------------------------------------------------------

\documentclass{article}

\input{structure.tex} % Include the file specifying the document structure and custom commands

%----------------------------------------------------------------------------------------
%	ASSIGNMENT INFORMATION
%----------------------------------------------------------------------------------------

\title{TSRMI: Assignment \#21} % Title of the assignment

\author{Luis Alberto Ballado Aradias\\ \texttt{luis.ballado@cinvestav.mx}} % Author name and email address

\date{CINVESTAV UNIDAD TAMAULIPAS --- \today} % University, school and/or department name(s) and a date

%----------------------------------------------------------------------------------------

\begin{document}

\maketitle % Print the title

%----------------------------------------------------------------------------------------
%	INTRODUCTION
%----------------------------------------------------------------------------------------

Describir brevemente c\'{o}mo se construye el modelo de un sensor ultras\'{o}nico y c\'{o}mo se le emplea para actualizar el conocimiento del mundo.\\

El modelo de un sensor ultrasónico se construye a partir de las características físicas y técnicas del sensor. Implica tener en cuenta factores como el patrón de emisión, la anchura del haz y las limitaciones de alcance. El modelo del sensor describe cómo se emiten las ondas ultrasónicas, se propagan por el entorno e interactúan con los objetos del espacio circundante.\\

Para actualizar el conocimiento del mundo mediante el sensor ultrasónico, el modelo del sensor se emplea en un proceso denominado fusión de sensores. Los datos del sensor, que incluyen las lecturas recibidas por el sensor ultrasónico, se combinan con los datos de otros sensores, como cámaras o lidar. El algoritmo de fusión de sensores integra la información de los distintos sensores para crear una percepción más precisa y completa del entorno circundante.\\

Utilizando el modelo del sensor ultrasónico, el algoritmo puede hacer estimaciones informadas sobre la ubicación, la forma y la distancia de los objetos detectados por el sensor ultrasónico. Este conocimiento actualizado del mundo puede utilizarse para diversas tareas robóticas, como la cartografía, la evitación de obstáculos o la localización en un entorno determinado. La combinación de la fusión de sensores y el modelo de sensores ayuda a mejorar la comprensión y el conocimiento del entorno por parte del robot.\\



\end{document}
