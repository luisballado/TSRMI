%%%%%%%%%%%%%%%%%%%%%%%%%%%%%%%%%%%%%%%%%
% Lachaise Assignment
% LaTeX Template
% Version 1.0 (26/6/2018)
%
% This template originates from:
% http://www.LaTeXTemplates.com
%
% Authors:
% Marion Lachaise & François Févotte
% Vel (vel@LaTeXTemplates.com)
%
% License:
% CC BY-NC-SA 3.0 (http://creativecommons.org/licenses/by-nc-sa/3.0/)
% 
%%%%%%%%%%%%%%%%%%%%%%%%%%%%%%%%%%%%%%%%%

%----------------------------------------------------------------------------------------
%	PACKAGES AND OTHER DOCUMENT CONFIGURATIONS
%----------------------------------------------------------------------------------------

\documentclass{article}

\input{structure.tex} % Include the file specifying the document structure and custom commands

%----------------------------------------------------------------------------------------
%	ASSIGNMENT INFORMATION
%----------------------------------------------------------------------------------------

\title{TSRMI: Assignment \#8} % Title of the assignment

\author{Luis Alberto Ballado Aradias\\ \texttt{luis.ballado@cinvestav.mx}} % Author name and email address

\date{CINVESTAV UNIDAD TAMAULIPAS --- \today} % University, school and/or department name(s) and a date

%----------------------------------------------------------------------------------------

\begin{document}

\maketitle % Print the title

%----------------------------------------------------------------------------------------
%	INTRODUCTION
%----------------------------------------------------------------------------------------

Documentar dos desarrollos cientificos recientes que realicen la construcción de un RRT para resolver algún problema de planificación, uno para robots manipuladores y otro para robots móviles.

\begin{itemize}

\item{¿Cuáles son las consideraciones particulares para cada una de estas implementaciones?} 

%\begin{figure}[h]
%\includegraphics[width=10cm]{images/vant.jpg}
%\centering
%\end{figure}

\item{¿Qué caracteristicas comparten?}

\item{¿Qué técnicas emplean para acelerar el proceso de búsqueda de soluciones?}

\item{¿Qué algoritmo utilizan para refinar la solución?}

\end{itemize}
  
\end{document}
