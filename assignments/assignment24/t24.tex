%%%%%%%%%%%%%%%%%%%%%%%%%%%%%%%%%%%%%%%%%
% Lachaise Assignment
% LaTeX Template
% Version 1.0 (26/6/2018)
%
% This template originates from:
% http://www.LaTeXTemplates.com
%
% Authors:
% Marion Lachaise & François Févotte
% Vel (vel@LaTeXTemplates.com)
%
% License:
% CC BY-NC-SA 3.0 (http://creativecommons.org/licenses/by-nc-sa/3.0/)
% 
%%%%%%%%%%%%%%%%%%%%%%%%%%%%%%%%%%%%%%%%%

%----------------------------------------------------------------------------------------
%	PACKAGES AND OTHER DOCUMENT CONFIGURATIONS
%----------------------------------------------------------------------------------------

\documentclass{article}

\input{structure.tex} % Include the file specifying the document structure and custom commands

%----------------------------------------------------------------------------------------
%	ASSIGNMENT INFORMATION
%----------------------------------------------------------------------------------------

\title{TSRMI: Assignment \#18} % Title of the assignment

\author{Luis Alberto Ballado Aradias\\ \texttt{luis.ballado@cinvestav.mx}} % Author name and email address

\date{CINVESTAV UNIDAD TAMAULIPAS --- \today} % University, school and/or department name(s) and a date

%----------------------------------------------------------------------------------------

\begin{document}

\maketitle % Print the title

%----------------------------------------------------------------------------------------
%	INTRODUCTION
%----------------------------------------------------------------------------------------

Inicialmente, el VANT se encuentra en tierra ($y_{0}$ = 0 y $y_{0}$ = 0). La aceleraci\'{o}n total sobre el VANT proporcionada por los motores es u = 0,1 m/$s^{2}$, con una desviación estándar de $\sigma_{u}$ = 0.1m/$s^{2}$ debida principalmente a la interacci\'{o}n entre el VANT y el suelo. El sensor de altitud (extremadamente barato) entrega lecturas cada d\'{e}cima de segundo, con una desviaci\'{o}n est\'{a}ndar $\sigma_{y}$ = 0.5m.\\

Implementar en Matlab la simulaci\'{o}n correspondiente a los primeros 10 segundos del vuelo del VANT, así como la estimación hecha por un filtro de Kalman. Analizar y reportar el efecto de la estimación y la covarianza iniciales sobre la evolución de la estimación proporcionada por el filtro. Analizar y reportar la sensibilidad del filtro a los cambios de desviación estándar $\sigma_{u}$ y $\sigma_{y}$.

\end{document}
