%%%%%%%%%%%%%%%%%%%%%%%%%%%%%%%%%%%%%%%%%
% Lachaise Assignment
% LaTeX Template
% Version 1.0 (26/6/2018)
%
% This template originates from:
% http://www.LaTeXTemplates.com
%
% Authors:
% Marion Lachaise & François Févotte
% Vel (vel@LaTeXTemplates.com)
%
% License:
% CC BY-NC-SA 3.0 (http://creativecommons.org/licenses/by-nc-sa/3.0/)
% 
%%%%%%%%%%%%%%%%%%%%%%%%%%%%%%%%%%%%%%%%%

%----------------------------------------------------------------------------------------
%	PACKAGES AND OTHER DOCUMENT CONFIGURATIONS
%----------------------------------------------------------------------------------------

\documentclass{article}

\input{structure.tex} % Include the file specifying the document structure and custom commands

%----------------------------------------------------------------------------------------
%	ASSIGNMENT INFORMATION
%----------------------------------------------------------------------------------------

\title{TSRMI: Assignment \#18} % Title of the assignment

\author{Luis Alberto Ballado Aradias\\ \texttt{luis.ballado@cinvestav.mx}} % Author name and email address

\date{CINVESTAV UNIDAD TAMAULIPAS --- \today} % University, school and/or department name(s) and a date

%----------------------------------------------------------------------------------------

\begin{document}

\maketitle % Print the title

%----------------------------------------------------------------------------------------
%	INTRODUCTION
%----------------------------------------------------------------------------------------

Un VANT navega en un medio ambiente a altura fija, recibiendo comandos de desplazamiento frontal, lateral y de giro sobre sí mismo, respecto a su propio sistema coordenado. Por perturbaciones del medio ambiente y otros fenómenos aleatorios, los desplazamientos del robot están contaminados con un ruido gaussiano (no medible) de media cero y varianzas $\sigma_{x}$, $\sigma_{y}$ y $\sigma_{\theta}$, para cada uno de los desplazamientos. Para ayudar a su localización, se colocaron 3 balizas dentro del ambiente en posiciones conocidas $r_{i}$ = ($x_{i}$ , $y_{i}$ ) con respecto a un marco de referencia fijo. El VANT cuenta con un sensor láser de espejo giratorio, que detecta el ángulo relativo al eje principal del robot en el que se encuentran las balizas reflejantes. El sistema de percepción es capaz también de identificar las balizas individualmente, por lo que la observación del sensor consiste en la tripleta de ángulos ($\beta_{1}$, $\beta_{2}$, $\beta_{3}$) sin importar el orden en que hayan sido detectadas las balizas. Cada ángulo tiene una imprecisión aleatoria de lectura, de distribución gaussiana de media cero y varianza $\sigma_{i}$. Implementar en Matlab la simulación correspondiente a los primeros 10 segundos del vuelo del VANT en algunas trayectorias predeterminadas, así como la estimación hecha por un filtro extendido de Kalman. Analizar y reportar el efecto de la estimación y la covarianza iniciales sobre la evolución de la estimación proporcionada por el filtro. Analizar y reportar la sensibilidad del filtro a los cambios de desviación estándar.



\end{document}
