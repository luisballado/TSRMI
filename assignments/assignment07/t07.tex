%%%%%%%%%%%%%%%%%%%%%%%%%%%%%%%%%%%%%%%%%
% Lachaise Assignment
% LaTeX Template
% Version 1.0 (26/6/2018)
%
% This template originates from:
% http://www.LaTeXTemplates.com
%
% Authors:
% Marion Lachaise & François Févotte
% Vel (vel@LaTeXTemplates.com)
%
% License:
% CC BY-NC-SA 3.0 (http://creativecommons.org/licenses/by-nc-sa/3.0/)
% 
%%%%%%%%%%%%%%%%%%%%%%%%%%%%%%%%%%%%%%%%%

%----------------------------------------------------------------------------------------
%	PACKAGES AND OTHER DOCUMENT CONFIGURATIONS
%----------------------------------------------------------------------------------------

\documentclass{article}

\input{structure.tex} % Include the file specifying the document structure and custom commands

%----------------------------------------------------------------------------------------
%	ASSIGNMENT INFORMATION
%----------------------------------------------------------------------------------------

\title{TSRMI: Assignment \#7} % Title of the assignment

\author{Luis Alberto Ballado Aradias\\ \texttt{luis.ballado@cinvestav.mx}} % Author name and email address

\date{CINVESTAV UNIDAD TAMAULIPAS --- \today} % University, school and/or department name(s) and a date

%----------------------------------------------------------------------------------------

\begin{document}

\maketitle % Print the title

%----------------------------------------------------------------------------------------
%	INTRODUCTION
%----------------------------------------------------------------------------------------

\section*{TAREA} % Unnumbered section
Para evitar calcular la forma exacta de la fontera entre el espacio libre y el espacio ocupado, algunos planificadores utilizan una versión discreta del C-Space 


%------------------------------------------------
\section{Documentar algunos desarrollos cientificos que empleen una representación discreta del espacio de configuraciones}


\begin{figure}[h]
\includegraphics[width=10cm]{images/drone_alafija.jpg}
\centering
\end{figure}

%------------------------------------------------
\newpage
\section{¿Cuáles son las diferencias, ventajas y desventajas entre la representación continua del C-Space y su correspondiente representación discreta?}

\newpage
\section{¿Cuáles son las diferencias entre esta representación y la representación de rejilla de ocupación (celdas fijas)?}

\end{document}
